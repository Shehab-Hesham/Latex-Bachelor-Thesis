%{
	Created by Shehab Hesham.
%}

\section{FPGAS}
\par	FPGAs are integrated circuits (ICs) designed to be user-configured according to the desired design requirements and characteristics, as defined by \cite{six}. The author provides several examples of image processing like image segmentation, feature extraction, and object recognition. Required times, time delays, memory restrictions, and resource constraints are also considered by the researcher in his publication. This research also explores the opportunities to interconnect the FPGA with suitable input sensors, camera modules, and display outputs and how the algorithm should be modified accordingly. \newline
\par There are always tradeoffs between performance and accuracy in any system. Researchers in \cite{one} explored the feasibility, challenges, performance, and accuracy of approximate FPGA design – more specifically approximate adders and multipliers – to implement a variety of color-to-greyscale converters. Results have shown improved power efficiency and performance when using approximate designs rather than the exact adder and multiplier models, this sacrifices some accuracy, but the results are still acceptable. The researchers came to the conclusion that approximate FPGA designs are better suited for image processing applications than exact FPGA modules. \newline
\par	FPGAs have an advantage over Digital Signal Processors (DSPs) because FPGAs are high-performance, low-power, low-cost, and flexible devices. In \cite{eight}, the authors highlight the complications an FPGA designer could face while implementing Image Processing (IP) algorithms; such as memory bandwidth limitations, design complications, and testbench design complexity. \cite{eight}’s researchers suggest that image processing on FPGAs is a field worth discovering and further researching, considering that the study’s results indicate that FPGAs perform better than DSPs in the tested algorithms: face recognition, edge detection, image compression, and image filtering. \newline
\par	Real-time operations are classified into three levels: hard, soft, and firm. One of the applications that cannot allow any time delays is space monitoring. SPACEMEDAL, by Northrop Grumman Corporation, is a small satellite for space meteor observation, using a Xilinx Zynq Ultrascale+MPSoC FPGA as the main Data Processing Unit (DSP). In \cite{eleven}’s evaluations, it has been recorded that speeds of up to 120 frames per second (fps) are able to be processed, consuming only 3.5 watts of power, this analysis surpassed the alternative software-based algorithm that processed only 10 fps and consumed almost double the power. Onboard image processing has been proved to reduce data volume by 99\% and the time required (latency) by more than 90\% when compared to sending the image data and implementing the IP algorithms on earth \cite{eleven}; moreover, the researchers reported that the design’s accuracy in object detection has increased by 20\% and were able to have an increased sensitivity in the system of +10 db.   \newline
\par	Resource Utilization, in FPGAs, aims to reduce the usage of logic elements, memory blocks, digital signal processor (DSP) slices, and I/O pins. Utilizing resources positively impacts a system’s performance, reduces power consumption, and reduces the system costs. An IP core is a reusable design excerpt able to be integrated and used in larger systems. Basic blocks of image algorithms include thresholding, negation, inversion, mirroring, brightness adjustment, and contrast adjustment. Researchers in \cite{fourteen} used IP cores to implement the aforesaid algorithms on an FPGA and tested the results against an equivalent software design, using an image stream captured from a camera sensor. The comparison between both hardware and software designs proved that the FPGA hardware design is faster and; hence, more efficient. \newline
\par	Many researchers aimed to compare the performances of DSPs and FPGAs; however, little to no research focused on integrating an FPGA chip with a DSP to provide the best performance possible. In this research, \cite{nineteen} provided an architectural design integrating an FPGA chip with a DSP processor., which they call TMS320DM642, this architecture was designed to perform edge detection and display the output through a VGA port. The assigned tasks were as follows: the FPGA had to sample the image and then display the processed image; while the DSP was responsible for the image processing tasks. In summary, a camera sends live frames to the FPGA, which samples the received frames, the sampled frames are sent over to the DSP for processing, and finally the frames are sent back to the FPGA, which displays the live stream in real-time through a VGA monitor. \newline

\subsection{Hardware Description Languages}
\par	Hardware Description Languages describe the structure and behavior of electronic circuits, while other programming languages, like C or Java, represent only software operations and sequential logic. Hardware Description Languages, Verilog and VHDL, are used to define Field Programmable Gate Arrays and Application Specific Integrated Circuits (FPGAs \& ASICs), this makes them flexible in design, but more complex. \cite{twelve} explores the IP acceleration possibility using FPGAs by implementing image processing algorithms by making use of an FPGA’s parallel computing capabilities. The researchers implemented image filtering and enhancement and edge detection algorithms, testing the three algorithms proved that there are significant speed increases and resource utilizations when the aforementioned algorithms are implemented on an FPGA rather than on a microcontroller. The authors also mentioned that their design can be implemented in video surveillance, robotics applications, and medical imaging. \newline
\par FPGA design is mainly ‘programmed’ using a hardware description language (HDL), such as VHDL or Verilog HDL; however, scientists of \cite{two} propose a software and hardware co-design that aims to convert the standard video format, PAL 576i, to standard Video Graphics Array format, VGA. The research also presents results demonstrating low resource utilization, flexibility, easy expansion, and adaptability to harsh and rugged environments. On top of the aforementioned core functionalities, the paper provides added features enabling the users to switch between numerous video streams to display, provides the option to add text to the video output, and provides an algorithm for skin color detection – which can be extended into a full human recognition algorithm. \newline
\par	As mentioned before, FPGAs are usually programmed using either Verilog HDL or VHDL; this is the main reason why FPGAs are not as widely used as industrially acknowledged microcontrollers. \cite{three} presents the “Single Assignment C” (SA-C), a special compiler used to design image processing algorithms for Field Programmable Gate Arrays (FPGAs). To verify the validity of the compiler design, the authors have implemented various modules using the newly implemented SA-C, these modules include scalar addition, convolution, histogram equalization, edge detection, motion estimation, color space conversion, image thresholding, and median filtering. In order to test whether the SA-C is a suitable compiler or not, the same algorithms were written in both assembly and C languages; the SA-C algorithms were tested on a Xilinx XV2000E FPGA, while the assembly and C codes were separately written on Intel’s Pentium III Processors. When all three programs ran at 800 MHz, the tested algorithms showed that the SA-C algorithms are 8 to 800 times faster (\cite{three} have not mentioned the results of the following algorithms: color space conversion, thresholding, and median filtering). \newline
\par	Verilog HDL is more commonly used when compared to VHDL used for designing digital systems on FPGAs and ASICs. The author of \cite{thirteen} implements image processing algorithms on FPGA using the MATLAB HDL Encoder while discussing design considerations, i.e., execution time and resource utilization. The researcher also provides a thorough and detailed code example to demonstrate how one can implement IP algorithms and this algorithm is then used for testing using a designed testbench. \newline
\par	Amongst the Hardware Description Languages available, Verilog Hardware Description Language is the most widely used hardware description language in industrial processes; unlike VHDL, which was mainly designed for educational purposes. Using Intel’s (Altera’s) Cyclone II FPGA, the researchers were able to design a Sobel edge detection algorithm and output the results on a VGA monitor in almost real-time.  The paper \cite{twenty} reports that the FPGA-implemented algorithm is 9 times faster than a software algorithm on microcontrollers, as the first uses parallel computing instead of the latter's serial computing technique. This algorithm, as claimed by the researchers, can be used as is in embedded industrial applications that cannot afford high energy consumption, nor huge time delays. \newline
\par	Verilog HDL and VHDL are the two most common languages used to program FPGAs; however, these are not the only available Hardware Description Languages, an overlooked language for FPGA programming is Reconfigurable Image Processing Language (RIPL), a Domain Specific Language (DSP). RIPL makes use of an FPGA's capability to perform parallelism and optimal memory efficiency to perform highly accurate Image Processing algorithms; RIPL is most known for easy syntax and simple implementation of IP algorithms. In this \cite{twenty_one} research, the authors focused on performing memory efficient and parallel performance real-time image processing algorithms. In terms of processing speed, resource utilization, and productivity, RIPL outperformed the more common Hard Description Languages (Verilog HDL \& VHDL), reported the researchers, when convolution, filtering, smoothing, image enhancements, upsampling an image, and horizontal splitting an image was considered. \newline

\section{Image Processing}
\par	Using Single Assignment C (SA-C) has proven to be faster than C or assembly algorithms; however, this is a relatively slow and ‘gate-consuming’ implementation. It is preferable to use a hardware description language, such as Verilog HDL, to directly design the desired algorithm on a Field Programmable Gate Array. A Hardware Description Language (HDL) is favored because most image-processing algorithms are usually designed to operate in hard real-time, making time a vital factor in one’s design. \cite{four} offers a direct implementation of useful image enhancement and manipulation filters in Verilog HDL to aid in image processing (IP) algorithms. Some of the filters include image sharpening, brightness enhancement, contrast adjustment, and edge detection. The authors claim that their designs have the advantage of rapid prototyping, as a Hardware Descriptive Language was directly used for the designs.  \newline
\par	Inaccurate computing is an energy-efficient solution to implement different algorithms, this method/technique can be used when a minor inaccuracy in the resulting value will not drastically deviate the output from the real output value but will result in a significant energy and/or time reduction. \cite{seven} emphasizes the benefits of using inaccurate computing techniques, rather than traditional and accurate techniques to implement simple algorithms, such as addition, multiplication, and digital image blending. Image blending can be defined as adjusting a masked image onto another image, in order to create a clearer visible view. This paper compares the results of accurate image blending algorithms and inaccurate image blending; the results are then compared and the quality of the blended images are evaluated. The analysis shows that an appropriate range of inaccuracy does not alter the resolution of an output image, but hugely impacts the energy consumption and the required time of the algorithm. \newline
\par	A CMOS image sensor uses an amplifier and a transistor for each pixel, in order to convert the captured light by each pixel into an electrical signal of a usable voltage output level, using a photodiode. The research conducted in \cite{nine} aimed to connect a CMOS image sensor to an FPGA, which will read image data and display the output on a VGA monitor. The involved researchers reported the success of their design that is claimed to be able to handle up to 50 frames per second (fps).  \newline
\par	Fast Fourier Transform (FFT) is a mathematical model used to convert a signal – either one-dimensional or two-dimensional - for example, an audio signal or image signal, from the time-space domain into the frequency domain. This conversion is done to allow more control over the processing algorithms, to enable easier implementation of various filters, considering that multiplication is used instead of convolution, and to ensure that the processing time is faster, as processing time is directly related to computational complexity. FFT is widely applicable in IP algorithms because dominant colors, edges, and textures can be easily determined and used in applications such as pattern recognition and feature extraction, another application is image reconstruction because, in the frequency domain, it is simpler to gather and put together different image sections. In \cite{ten}, the researchers used the FFT for image compression and encryption. One can implement the Fast Fourier Transform, such as radix-2, radix-4, split-radix, and the Fast Harley Transform (FHT) method. The researchers claimed that by implementing the FFT better results can be reached, as processing can be done in almost real-time, ensuring high performances.\cite{ten} has also provided a set of filters, such as high-pass, band-pass, and notch filters. \newline
\par	Image thresholding is done by setting a number, a threshold value, where any pixels that carry values smaller than the threshold are assigned to 0, a black color, and any pixels holding a value larger than the set value are assigned the white color, a value of 255 or 1. Image Thresholding can be used to make some desirable features clearer and unwanted features more invisible. Different parts of different images have different lighting and contrast conditions, so setting a fixed threshold value and using the traditional thresholding technique is not always ideal; a good alternative would be to use an adaptive thresholding technique. Researchers in \cite{fifteen} have implemented an algorithm for Adaptive Canny Edge Detection that was tested on a 3E FPGA kit and used a 5MP camera taking an input image stream, after the edge detection algorithm is performed, the manipulated images are displayed on a VGA monitor. The Canny Edge Detection was divided into four subsections; Gaussian filtering, gradient/direction calculation, non-maximal suppression, and thresholding/hysteresis.  As expected, using an FPGA allowed the use of an improved edge detection mechanism, thus increasing the efficiency, while maintaining the time restrictions. \newline
\par	Edge Detection can be implemented in various ways like the Canny detection method, Sobel detection method, and Prewitt detection method. Algorithms can be implemented in software languages, i.e., C or Java, in hardware description languages, like Verilog HDL or VHDL, or in a Model-based approach. Authors of \cite{sixteen} reported the results of their approaches to implementing the Sobel and Prewitt filters in both a Hardware Description Language and a Model-Based approach. The research provides results proving that FPGA implementations of the mentioned algorithms outperformed the Model-based implementations when the researchers compared them in accuracy, resource utilization, and speed, making the FPGA implementation faster and more efficient. \newline
\par	There are different techniques to implement the same functions, this also applies to edge-detection. The most common edge detection techniques are the Canny edge detection method, the Sobel edge detection method, and the Prewitt edge detection method. However, there is a method called gradient-based edge detection, this is considered by many a category of edge detection based on the derivative of the image, which includes Sobel and Prewitt edge detection techniques. \cite{eighteen} presents a technique to implement an optimized gradient-based edge detection method that reduces computational time by 50\%. A Spartan 3E FPGA board was interfaced with a VGA monitor to process and display the images, respectively. \newline
\par	XIL XSGlmgLib is a generic library containing architectures that can be self-modified to fit a certain process’s needs, this library contains configurable hardware modules designed to aid in image and video processing on FPGAs. Using a Spartan-6 LX45 FPGA, \cite{twenty_six} aimed to discuss features of the mentioned library and demonstrate ways to use it. This library was designed to aid in the development of image processing-related applications, allowing easier and faster development, scalability, and monitoring. \newline

\section{Video Graphics Array}
\par	Apart from what algorithms are done to the input video, displaying an output to a screen requires some work. A Video Graphics Array (VGA) is one way to display an image or pattern on a monitor; however, this requires the design of a VGA controller, which will function based on two clock signals, a frequency for the horizontal lines, and one responsible for writing the values on each vertical line. The authors of \cite{five} a design to display a video stream on a monitor using a VGA output port. The paper suggests that the design provided is the most reliable in rugged applications and is the most suited for further expansion because it utilizes substantially fewer resources. \newline
\par	Video Graphics Array (VGA), High-Definition Multimedia Interface (HDMI), Digital Visual Interface (DVI), Display Port, USB-C, Thunderbolt, RCA, NDI, and SDI are all means of displaying an image; the most convenient, however, is the VGA port display. Video Graphics Array uses a graphics processor and video memory to store pixel data and send the analog signals, converted through an R-2R Digital to Analog Converter (DAC), through a cable. VGA output mainly requires 2 clock signals: HSYNCH AND VSYNCH, responsible for indicating a new horizontal line is ready to be scanned and indicating that a new frame is starting respectively, these clock signals are used to ensure correct image display with minimal distortion. In this study \cite{twenty_three}, the author used Xilinx’s Vivado Design Suite to study a VGA system’s timing characteristics and use the knowledge to display different patterns of color on a VGA monitor, using a Nexys-4 DDR FPGA board. The researcher also provides the simulation results of the signal generation, synchronization, character and symbol displaying, and color encoding algorithms. \newline

\section{Parallelism}
\par	Parallelism is the term describing when more than one process is being executed at the same time. Field Programmable Gate Arrays (FPGAs) have the capability of performing tasks concurrently, rather than sequentially. This thesis, \cite{seventeen} provides a design implementing image compression, image enhancement, and edge detection, while comparing the performance to a PC software implementation. This paper, unlike other resources, made use of the MATLAB HDL Encoder, not the traditional Xilinx Vivado Suite and Intel’s Altera Quartus Prime. The researcher reported that the FPGA implementation exceeded the speed of the software design due to its parallelism and reconfigurability. \newline

\section{Applications}
\par	Internet of Things (IoT) is a growing field of study, as all industries are aiming to implement Industrial Internet of Things (IIoT) concepts in their field of work. Industrial Internet of Things-based industries are required to gather many different sensors’ and cameras’ data, process this data, and then take appropriate actions and fire the appropriate alarms, all while keeping a memory log of all data. These tasks require fast computation, as actions have to be in real-time, as many safety-critical processes are in store; thus, risking human lives.  For this reason, researchers in \cite{twenty_two} proposed a design where Field Programmable Gate Arrays (FPGAs) are used for image processing, one of the tasks required in an Industrial Internet of Things-based industrial process. The proposed design is an optimized, internet-accessible, pipelined, and low-cost algorithm performing image mirroring, inversion, negation, thresholding, and brightness and contrast adjustment. The system integrates a Spartan 6 FPGA with a Raspberry B+, which is used as an ARM cortex-M-based System on Chip (SoC). The researchers also compared the developed system in terms of computation time, time delays, and chip resource utilization to a standard software-implemented image processing tool; the proposed integrated design has proven to be a good, low-cost alternative in various fields, including medical applications, space observation, and IIoT. \newline
\par	Aerospace applications such as meteor detection or planet tracking require substantial accuracy and tremendous speeds; however, at such altitudes and in these harsh conditions, it is hard to find suitable systems that can withstand such rugged environments. With FPGAs’ ability to concurrently process data and withstand harsh environments, to some extent, it is worth studying the difficulties faced in meteor detection systems and study whether or not an FPGA can be mounted into such a system to improve the performance of the overall system. Meteors are in constant fast motion, which requires processing units to compute algorithms as fast as possible, to be able to cope with the high angular velocity changes of a meteor. Another major hindering factor is that background noise in an image containing a meteor-like body has many light sources (stars and other planets) that act as noise, making it harder to differentiate a meteor’s trial from a noisy source. In \cite{twenty_four}, the scientist discusses the challenges faced and possible solutions to meteor imagery systems using different instruments; some mentioned electro-optical instruments discussed are all-sky imagers, wide-field intensified video, and narrow field-of-view telescopic systems. For the fast-moving cluster detection algorithm required for such a system, Hough transforms can be used to assist in meteor streak detection. Hough transform can be optimally implemented using a pipelined method, thus offering faster speeds and more accurate results. \newline
\par	Low latency, high performance, and lower costs can be achieved by using an FPGA – such as the Spartan 3E FPGA – to implement image processing algorithms, instead of the traditional software-based implementations. In \cite{twenty_five}, images were displayed through a VGA-interfaced monitor after the Canny edge detection algorithm was implemented. To implement the Canny edge detection method, one has to implement Gaussian filtering, gradient and direction calculation, non-maximal suppression, and thresholding and hysteresis; however, these complex algorithms consume lots of power and require a lot of time to compute, which is not always a luxury in a process. To overcome these problems, researchers in \cite{twenty_five} used approximate models instead of exact models, in order to save resources, time, and power. This method is claimed to be a low-cost, high-performance method for edge detection, output images of size 128x128 were viewed on a VGA monitor and compared to results of MATLAB simulations. \newline



\medskip

%\printbibliography
%\end{document}