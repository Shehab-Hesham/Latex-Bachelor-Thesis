%{
	Created by Shehab Hesham.
%}

\chapter*{Abstract}

Medical applications, aerospace models, and automotive systems are a few of the many systems that require almost real-time signal processing and near-instant decision-making. Many industries have applied Industrial Internet of Things (IIoT) methodologies, requiring countless sensors, cameras, controllers, monitors, and actuators to communicate together to gather data, save a data log, process data, display data on monitors, take appropriate actions, and fire necessary alarms all at the same time. Nowadays, most processing applications make use of a traditional microcontroller or a Programmable Logic Controller (PLC) to perform necessary algorithms for different applications. Field Programmable Gate Arrays (FPGAs) are devices that mainly consist of configurable logic blocks. These logic blocks implement sequential and combinational logic using flip-flops and look-up tables. The main advantages of an FPGA are flexibility in design and parallel computation, which allows the FPGA to concurrently perform tasks. This decreases the time required to process data, whether from sensors or cameras, and send appropriate commands to take necessary actions and activate the appropriate alarms. However, Field Programmable Gate Arrays are strictly programmed using a Hardware Description Language, either Verilog HDL or VHDL, making FPGAs complex in programming; hence, uncommon in industrial applications. Recently, FPGAs have been introduced in some minor applications for image processing used in automotive applications, but this is not the most an FPGA can do. For this reason, this study aims to explore the different image processing algorithms that can be implemented efficiently on an FPGA. Moreover, an image library is designed to manipulate images and output the manipulated image using a Video Graphics Array format.           \newline

\textbf{Keywords:} FPGA, Image Processing, Image Library, Video Graphics, Verilog HDL




%\end{document}
